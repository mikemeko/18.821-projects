% Staircase paper for 18.821 Fall 2012 project 3.
% Authors: Daniel Grazian, Michael Mekonnen, Agustin O Venezuela III

\documentclass[12pt]{amsart}

% Keep everything here in alphabetical order, please! :) -jven

% Packages

\usepackage{amssymb}

% Enumeration

\newtheorem{theorem}{Theorem}[section]

\newtheorem{conjecture}[theorem]{Conjecture}
\newtheorem{corollary}[theorem]{Corollary}
\newtheorem{definition}[theorem]{Definition}
\newtheorem{example}[theorem]{Example}
\newtheorem{examples}[theorem]{Examples}
\newtheorem{lemma}[theorem]{Lemma}
\newtheorem{proposition}[theorem]{Proposition}
\newtheorem{remarks}[theorem]{Remarks}
\newtheorem{remark}[theorem]{Remark}

% Utility commands
% Real numbers
\newcommand{\R}{\mathbb{R}}
% Figures: \newfigure{label}{caption}{content}
\newcommand{\newfigure}[3]{
\begin{figure}
#3
\caption{#2 \label{#1}}
\end{figure}
}
% Sections: \newsection{title}{label}
\newcommand{\newsection}[2]{
\section{#1 \label{#2}}
}

\title{A Staircase Model of Erosion}
\author{Daniel Grazian, Michael Mekonnen, Agustin O Venezuela III}
\date{November 3, 2012}

\begin{document}

\begin{abstract}
TODO(dgrazian, jven, mikemeko)
\end{abstract}

\maketitle

\newsection{Introduction}{sec:intro}
In this paper, we consider the random process of forming staircases by dropping blocks into an infinite row of infinitely tall columns.

More concretely, consider partitioning the first quadrant of $\R^2$ into axis-aligned columns of width $1$, as in Figure $\ref{fig:columns}$. We will index the columns starting at $0$.

\newfigure{fig:columns}{Partitioning of the first quadrant into columns}{
BLAH
}

Now consider iteratively placing unit squares (blocks) into these columns. We will say that a block is at location $(i, j)$ for non-negative integers $i, j$ if it is axis-aligned and its bottom-left vertex is at $(i, j)$. At each step, a block can be placed at $(i, j)$ if and only if (1) $i = 0$ or there is a block at $(i - 1, j)$ and (2) $j = 0$ or there is a block at $(i, j - 1)$. Figure $\ref{fig:dropblocks}$ shows an example of a valid sequence of placing $5$ blocks. We will usually refer to such a sequence of placements a \textit{dropping of $5$ blocks}, for obvious reasons. We call a configuration of $n$ blocks in the first quadrant a \textit{staircase} if it can be obtained by dropping $n$ blocks.

\newfigure{fig:dropblocks}{Example of dropping $5$ blocks}{
SUP
}

We will refer to a staircase as the monotonically decreasing finite sequence of positive integers $(b_i)$, where $b_i$ is the number of blocks in column $i$. For example, the last staircase in Figure \ref{fig:dropblocks} can be written as $(hi,hi,hi,hi,hi)$. Note that we do not include columns that do not contain any blocks.

We are interested in constructing random staircases: beginning with no blocks, we consider all the locations $(i, j)$ at which a block can be validly placed, choose such a location uniformly at random, and place a block there. This raises various interesting questions regarding the distribution over the shape of the resulting staircase when $n$ blocks are dropped.

To begin, Section \ref{sec:numstaircases} will address the question of how many staircases exist with $n$ blocks. Section \ref{sec:expectedcolumns} will present numerical results for the expected number of columns for a staircase with $n$ blocks. Finally, Section \ref{sec:twocolumn} will present exact results for the variant of the problem in which we restrict staircases to having at most $2$ columns.

\newsection{Number of Staircases with $n$ Blocks}{sec:numstaircases}
Our investigation into the distribution over staircase shapes begins with determining how many staircase shapes exist using $n$ blocks.

\begin{theorem}
The number of staircases with $n$ blocks is equal to the number of partitions of $n$ (the number of ways to write $n$ as a sum of positive integers, irrespective of the order of the parts).
\begin{proof}
There is an obvious bijection $f$ between staircase with $n$ blocks and partitions of $n$. We map the staircase $(b_i)$ with $n$ blocks to the partition $n=\sum b_i$.

$f$ is injective. If $(b_i)\neq (b_i')$ are two distinct staircases, then $n = \sum b_i = \sum b_i'$ are two partitions of $n$ such that with the parts written in monotonically decreasing order, we have some $b_i\neq b_i'$. It follows that the two partitions are distinct.

$f$ is also surjective. Given a partition $n = \sum a_i$, we can sort $(a_i)$ in monotonically decreasing order, yielding $(b_i)$. $f$ clearly maps $(b_i)$ to the partition $\sum a_i$.

The result follows.
\end{proof}
\end{theorem}

For increasing $n$ beginning with $n = 0$, the numbers of partitions of $n$ are $1, 1, 2, 3, 5, 7, 11, \ldots$. Unfortunately, this sequence is known to grow exponentially with $n$: for $n = 100$, there are $190,569,292$ partitions. This appears to make our original goal of finding a distribution over staircase shapes difficult as there are exponentially many objects to which a probability is to be assigned. For this reason, we instead consider the distribution over the number of columns in the next section, the number of which is clearly linear in $n$.

\newsection{Expected Number of Columns}{sec:expectedcolumns}
TODO(dgrazian)

\newsection{2-Column Staircases}{sec:twocolumn}
TODO(mikemeko)

\end{document}