% beamer_template
% Created by Christopher Schommer-Pries on 2012-02-16.
% Copyright (c) 2012. All rights reserved.

% Graphing fleas slides for 18.821 Fall 2012 project 1.
% Authors: Daniel Grazian, Michael Mekonnen, Agustin O Venezuela III

\documentclass{beamer}		%% The Beamer document class formats for slides. 

\usetheme{CambridgeUS} 		%% Formats the slides with a red and gray theme, which seems appropriate for MIT.  

% Keep everything here in alphabetical order, please! :) -jven

% Packages

\usepackage{amsfonts}
\usepackage{amssymb}
\usepackage{graphicx}	

% Enumeration

\newtheorem{proposition}[theorem]{Proposition}
\newtheorem{remarks}[theorem]{Remarks}
\newtheorem{remark}[theorem]{Remark}
\newtheorem{conjecture}[theorem]{Conjecture}

% Utility commands

% Set of all binary sequences
\newcommand{\binseq}{\{0,1\}^*}
% Set of first n non-negative numbers
\newcommand{\firstn}[1]{\ensuremath{\{0, \ldots, #1 - 1\}}}
% Left arrow accent (the negative hspace is required because
% the over arrows are massive)
\newcommand{\laa}[1]{\hspace{-0.11cm}\overleftarrow{#1}}
% Right arrow accent 
\newcommand{\raa}[1]{\hspace{-0.11cm}\overrightarrow{#1}}
% 2-complete rule
\newcommand{\rtwo}{R}
% World column (e.g. ...[0]0|00...)
\newcommand{\wc}{p{0cm}}
% World column with negative numbers
\newcommand{\wcc}{p{.3cm}}
% Evolution of initial world under rtwo
\newcommand{\wtwo}{E(R)}
% Integers
\newcommand{\z}{\mathbb{Z}}
% Non-negative integers
\newcommand{\znn}{\mathbb{N}}
% Positive integers
\newcommand{\zp}{\mathbb{Z}^+}

\title{Graphing Fleas}
\author[Grazian, Mekonnen, Venezuela]{Daniel Grazian, Michael Mekonnen, Justin Venezuela}
\institute[MIT]{Massachusetts Institute of Technology}
\date{October $22^\text{nd}$, 2012}

\begin{document}

% Daniel

\frame{
\titlepage
}

\frame{
\frametitle{Discrete Number Line With States and Flea}
$$\begin{array}{c|c|c|c|c|c|c|c|c}
\ldots & \phantom{0} & \phantom{0} & \phantom{0} & \phantom{0} & \phantom{0} & \phantom{0} & \phantom{0} & \ldots
\end{array}$$

}

\frame{
\frametitle{Discrete Number Line With States and Flea}
$$\begin{array}{c|c|c|c|c|c|c|c|c}
\ldots & 1 & 0 & 2 & 1 & 0 & 0 & 1 & \ldots
\end{array}$$
}

\frame{
\frametitle{Discrete Number Line With States and Flea}
$$\begin{array}{r|r|r|r|r|r|r|r|r}
\ldots & 1 & 0 & 2 & \raa{1} & 0 & 0 & 1 & \ldots
\end{array}$$
}

\frame{
\frametitle{Discrete Number Line With States and Flea}
$$\begin{array}{r|r|r|r|r|r|r|r|r}
\ldots & 1 & 0 & 2 & 1 & \raa{0} & 0 & 1 & \ldots
\end{array}$$
}





\frame{
\frametitle{Rule determines behavior}
\pause
State of point where flea is located determines:
\pause
\begin{enumerate}
\item what state the point changes to when the flea steps off.
\item whether the flea changes direction before stepping off.

\pause

\end{enumerate}

\begin{itemize}
\item Many possible rules

\item We focus on particularly interesting ones.

\item Can a rule produce every sequence of states?
\end{itemize}
}

\frame{
\frametitle{Rule determines behavior}
\begin{itemize}
\item A rule consists of a series of statements of the form:

\emph{`When the flea steps off a point in state $X$, the point changes to state $Y$, and the flea does/doesn't reverse direction before stepping off.'}
\pause

\item One such statement for every possible state.
\pause
\item $(2k)^{k}$ possible rules, where $k$ is the number of allowed states.
\end{itemize}
}

\frame{
\frametitle{Rule determines behavior}
$$\begin{array}{c|c|c|c|c|c|c|c|c}
\ldots & 1 & 0 & 1 & \raa{1} & 0 & 0 & 1 & \ldots
\end{array}$$
\pause
$$\begin{array}{c|c|c|c|c|c|c|c|c}
\ldots & 1 & 0 & 1 & 1 & \raa{0} & 0 & 1 & \ldots
\end{array}$$

`When the flea steps off a point in state $1$, the point changes to state 1, and the flea doesn't reverse direction before stepping off.'

\pause

$$\begin{array}{c|c|c|c|c|c|c|c|c}
\ldots & 1 & 0 & 1 & \laa{1} & 1 & 0 & 1 & \ldots
\end{array}$$

`When the flea steps off a point in state $0$, the point changes to state 1, and the flea reverses direction before stepping off.'

}

\frame{
\frametitle{Rule determines behavior}
Formally, a rule is a function from state to state and direction change
\pause

\vspace{.3cm}

$r: \{0,\ldots,k-1\} \to \left(\{0,\ldots,k-1\}, \{-1,1\}\right)$
\vspace{.3cm}

where $k$ is the number of allowed states.

\pause
\vspace{.3cm}

\begin {example}
$$
r: \begin{array}{ccc}
0 & \to & (1,-1)\\
1 & \to & (1, 1)\\
\end{array}
$$
\end{example}
}

\frame{
\frametitle{Definitions and Notation}
\pause

\begin {itemize}
\item A \emph{world} describes the state at every point and gives the location and direction of the flea.

$$\begin{array}{llrrrrllllll}
W_0: & \ldots & 0 & 0 & 0 & 0 & | & \raa{0} & 0 & 0 & 0 & \dots
\pause
\end{array}$$

\vspace{.2cm}

\item The \emph{evolution} $E\left(r\right)$ of a rule $r$ is the infinite sequence $\left(W_0, W_1, W_2, \ldots\right)$ of worlds produced by iteratively applying $r$ to $W_0$:
\pause

\vspace{.2cm}

\item A world \emph{contains} a finite sequence $\left(s_0, s_1, \ldots, s_{n-1}\right)$ of states if points $0, 1, \ldots, n-1$ are in states $s_0, s_1, \ldots, s_{n-1}$ respectively.
\pause

\vspace{.2cm}

\item An evolution E(r) \emph{accepts} a sequence of states if any world in E(r) contains the sequence.

\end{itemize}
}

\frame{
\frametitle{Example}
\begin{example}

$$
r: \begin{array}{ccc}
0 & \to & (1,-1)\\
1 & \to & (1, 1)\\
\end{array}
$$

\pause

$$\begin{array}{llrrrrllllll}
W_0 & \ldots & 0 & 0 & 0 & 0 & | & \raa{0} & 0 & 0 & 0 & \dots\\
W_ 1 & \ldots & 0 & 0 & 0 & \laa{0} & | & 1 & 0 & 0 & 0 &\ldots\\
\pause
W_ 2 & \ldots & 0 & 0 & 0 & 1 & | & \raa{1} & 0 & 0 & 0 &\ldots\\
W_ 3 & \ldots & 0 & 0 & 0 & 1 & | & 1 & \raa{0} & 0 & 0 &\ldots
\pause
\end{array}$$
\end{example}
\begin{itemize}
\item The first four worlds of the \emph{evolution} E(r) of $r$ are shown above.
\item $W_2$ \emph{contains} $(1,0)$ and $(1,0,0)$
\item Therefore E(r) \emph{accepts} $(1,0)$ and $(1,0,0)$.
\end{itemize}

}

\frame{
\frametitle{A $k$-complete Rule}
\pause

A rule $r$ is \emph{k-complete} if the evolution of $r$ accepts every finite sequence of the states $0, \ldots, k-1$.
\pause

\vspace{.5cm}

For example, a 2-complete rule must accept:
\pause

\vspace{.5cm}

$(0)$

$(1)$

\pause

$(0,1)$

$(1,0)$

$(0,1)$

$(1,1)$

\pause

$(0,0,0)$

\ldots

$(1,1,1)$

\ldots

}



\frame{
\frametitle{Our Main Result}

\pause

\begin{theorem}
We can construct a $k$-complete rule over $k+1$ states for all $k\in \zp$.
\end{theorem}

\pause

In this talk, we'll give a $2$-complete rule over $3$ states.

\pause

\vspace{.7cm}

We will then show how `fast' this rule is in generating every possible sequence of 0's and 1's.
}

% Justin

\frame{
\frametitle{Example of a $2$-complete rule}

\pause

$$\rtwo: \begin{array}{ccc}
0 & \to & (1, -1) \\
1 & \to & (2, 1) \\
2 & \to & (0, 1)
\end{array}$$

}

\frame{
\frametitle{First few worlds of $\wtwo$}

$$\rtwo: \begin{array}{ccc}
0 & \to & (1, -1) \\
1 & \to & (2, 1) \\
2 & \to & (0, 1)
\end{array}$$

\pause

$$\begin{array}{ccccccccc}
\ldots & 0 & 0 & 0 & | & \raa{0} & 0 & 0 & \ldots \\ \pause
\ldots & 0 & 0 & \laa{0} & | & 1 & 0 & 0 & \ldots \\ \pause
\ldots & 0 & 0 & 1 & | & \raa{1} & 0 & 0 & \ldots \\ \pause
\ldots & 0 & 0 & 1 & | & 2 & \raa{0} & 0 & \ldots \\ \pause
\ldots & 0 & 0 & 1 & | & \laa{2} & 1 & 0 & \ldots \\ \pause
\ldots & 0 & 0 & \laa{1} & | & 0 & 1 & 0 & \ldots
\end{array}$$
}

\frame{
\frametitle{Demo!}
}

\frame{
\frametitle{Claim}

\pause

\begin{theorem}
$\rtwo$ is $2$-complete.
\end{theorem}

\pause

Key observation:

\alert{$\wtwo$ accepts all sequences of length $n$ before the flea leaves $[-n,n-1]$.}

}

\frame{
\frametitle{$\wtwo$ in $[-1,0]$}

\pause

$$\begin{array}{ccccc}
\ldots & 0 & | & \raa{0} & \ldots \\ \pause
\ldots & \laa{0} & | & 1 & \ldots \\ \pause
\ldots & 1 & | & \raa{1} & \ldots
\end{array}$$

}

\frame{
\frametitle{$\wtwo$ in $[-2,1]$}

\pause

$$\begin{array}{ccccccc}
\ldots & 0 & \textbf{0} & | & \raa{\textbf{0}} & 0 & \ldots \\ \pause
\downarrow & & & & & & \\
\ldots & 0 & \textbf{1} & | & \raa{\textbf{1}} & 0 & \ldots \\ \pause
\ldots & 0 & 1 & | & 2 & \raa{0} & \ldots \\ \pause
\ldots & 0 & 1 & | & \laa{2} & 1 & \ldots \\ \pause
\ldots & 0 & \laa{1} & | & 0 & 1 & \ldots \\ \pause
\ldots & \laa{0} & 2 & | & 0 & 1 & \ldots \\ \pause
\ldots & 1 & \raa{2} & | & 0 & 1 & \ldots \\ \pause
\ldots & 1 & \textbf{0} & | & \raa{\textbf{0}} & 1 & \ldots \\ \pause
\downarrow & & & & & & \\
\ldots & 1 & \textbf{1} & | & \raa{\textbf{1}} & 1 & \ldots \\ \pause
\ldots & 1 & 1 & | & 2 & \raa{1} & \ldots
\end{array}$$

}

\frame{
\frametitle{$\wtwo$ in $[-3,2]$}

\pause

$$\begin{array}{ccccccccc}
\ldots & 0 & \textbf{0} & \textbf{0} & | & \raa{\textbf{0}} & \textbf{0} & 0 & \ldots \\ \pause
\downarrow & & & & & & & & \\
\ldots & 0 & \textbf{1} & \textbf{1} & | & \textbf{2} & \raa{\textbf{1}} & 0 & \ldots \\ \pause
\ldots & 0 & 1 & 1 & | & 2 & 2 & \raa{0} & \ldots \\ \pause
\downarrow & & & & & & & & \\
\ldots & \laa{0} & 2 & 2 & | & 0 & 0 & 1 & \ldots \\ \pause
\downarrow & & & & & & & & \\
\ldots & 1 & \textbf{0} & \textbf{0} & | & \raa{\textbf{0}} & \textbf{0} & 1 & \ldots \\ \pause
\downarrow & & & & & & & & \\
\ldots & 1 & \textbf{1} & \textbf{1} & | & \textbf{2} & \raa{\textbf{1}} & 1 & \ldots \\ \pause
\ldots & 1 & 1 & 1 & | & 2 & 2 & \raa{1} & \ldots
\end{array}$$

}

\frame{
\frametitle{Proof of $2$-completeness}

\pause

By induction. \pause Induction hypothesis:

\begin{itemize}
\item $0^n \raa{0}0^{n-1} \to 1^n2^{n-1}\raa{1}$ \pause
\item ... while staying within these columns \pause
\item ... while accepting all sequences of length $n$ \pause $\qed$
\end{itemize}

\pause

\begin{corollary}
We can similarly define the notions of \underline{negative containment} and \underline{negative acceptance}.

\vspace{0.2in}
\pause

Every sequence accepted by $\wtwo$ is also negatively contained.
\end{corollary}

}

% Michael

\frame{
\frametitle{$\rtwo$ is 2-complete}

$E(\rtwo)$ accepts and negatively accepts every finite binary sequence.

}

\frame{
\frametitle{$\rtwo$ is 2-complete}
\begin{example}

$b = 01$ \pause

$$\begin{array}{ccccccccc}
\ldots & 0 & 0 & 0 & | & \raa{0} & 0 & 0 & \ldots \\ \pause
\ldots & 0 & 0 & \laa{0} & | & 1 & 0 & 0 & \ldots \\ \pause
\ldots & 0 & 0 & 1 & | & \raa{1} & 0 & 0 & \ldots \\ \pause
\ldots & 0 & 0 & 1 & | & 2 & \raa{0} & 0 & \ldots \\ \pause
\ldots & 0 & 0 & 1 & | & \laa{2} & 1 & 0 & \ldots \\ \pause
\ldots & 0 & 0 & \laa{1} & | & \alert{\textbf{0}} & \alert{\textbf{1}} & 0 & \ldots \\ \pause
\ldots & 0 & \laa{0} & 2 & | & 0 & 1 & 0 & \ldots \\ \pause
\ldots & 0 & 1 & \raa{2} & | & 0 & 1 & 0 & \ldots \\ \pause
\ldots & 0 & \alert{\textbf{1}} & \alert{\textbf{0}} & | & \raa{0} & 1 & 0 & \ldots
\end{array}$$

\end{example}
}

\frame{
\frametitle{How fast is $\rtwo$?}
After how many worlds do we see all binary sequences of length $n$?

\pause
\vspace{.5in}

Surely \textbf{$\Omega(2^n)$}.
}

\frame{
\frametitle{How fast is $\rtwo$?}
\begin{definition}
\alert{$T_n$}: minimum number worlds to see all finite binary sequences of length $n$.
\end{definition}

}

\frame{
\frametitle{$T_1$}
\begin{example}

$T_1 =$ ?

Looking for $0$ and $1$.

\end{example}
}

\frame{
\frametitle{$T_1$}

\begin{example}

\begin{columns}
\column{.85\textwidth}
$$\begin{array}{ccccccccc}
\ldots & 0 & 0 & \alert{\textbf{0}} & | & \raa{\alert{\textbf{0}}} & 0 & 0 & \ldots \\ \pause
\ldots & 0 & 0 & \laa{0} & | & \alert{\textbf{1}} & 0 & 0 & \ldots \\ \pause
\ldots & 0 & 0 & \alert{\textbf{1}} & | & \raa{1} & 0 & 0 & \ldots
\end{array}$$

\column{.15\textwidth}
\alert{$T_1 = 2$}
\end{columns}

\end{example}

}

\frame{
\frametitle{$T_2$}
\begin{example}

$T_2 =$ ?

Looking for $00$, $01$, $10$, and $11$.

\end{example}
}

\frame{
\frametitle{$T_2$}

\begin{example}

\begin{columns}
\column{.85\textwidth}

$$\begin{array}{ccccccccc}
\ldots & 0 & \alert{\textbf{0}} & \alert{\textbf{0}} & | & \raa{\alert{\textbf{0}}} & \alert{\textbf{0}} & 0 & \ldots \\ \pause
\ldots & 0 & 0 & \laa{0} & | & \alert{\textbf{1}} & \alert{\textbf{0}} & 0 & \ldots \\ \pause
\ldots & 0 & \alert{\textbf{0}} & \alert{\textbf{1}} & | & \raa{1} & 0 & 0 & \ldots \\ \pause
\ldots & 0 & 0 & 1 & | & 2 & \raa{0} & 0 & \ldots \\ \pause
\ldots & 0 & 0 & 1 & | & \laa{2} & 1 & 0 & \ldots \\ \pause
\ldots & 0 & 0 & \laa{1} & | & \alert{\textbf{0}} & \alert{\textbf{1}} & 0 & \ldots \\ \pause
\ldots & 0 & \laa{0} & 2 & | & 0 & 1 & 0 & \ldots \\ \pause
\ldots & 0 & 1 & \raa{2} & | & 0 & 1 & 0 & \ldots \\ \pause
\ldots & 0 & \alert{\textbf{1}} & \alert{\textbf{0}} & | & \raa{0} & 1 & 0 & \ldots \\ \pause
\ldots & 0 & 1 & \laa{0} & | & \alert{\textbf{1}} & \alert{\textbf{1}} & 0 & \ldots \\ \pause
\ldots & 0 & \alert{\textbf{1}} & \alert{\textbf{1}} & | & \raa{1} & 1 & 0 & \ldots
\end{array}$$

\column{.15\textwidth}
\alert{$T_2 = 10$}
\end{columns}

\end{example}

}

\frame{
\frametitle{$T_n$}

\textbf{Goal:} a recurrence for $T_n$.

\pause
\vspace{.5in}

\textbf{Method:} study the progression from $W_0$ to $W_{T_n}$ in $\wtwo$.

}

\frame{
\frametitle{Describing $W_{T_n}$}

\begin{lemma}
\begin{enumerate}
\item The world $W_{T_n}$ has the form: $\ldots 0 1^{n} | \raa{1} 1^{n-1} 0 \ldots$ \pause
\item The progression from $W_0$ to $W_{T_n}$ does not step outside of the locations between $-n$ and $n-1$.
\end{enumerate}
\end{lemma}
}

\frame{
\frametitle{$W_{T_1}$}

$$\begin{array}{ccccccccc}
\ldots & 0 & 0 & 0 & | & \raa{0} & 0 & 0 & \ldots \\
\ldots & 0 & 0 & \laa{0} & | & 1 & 0 & 0 & \ldots \\
\alert{\ldots} & \alert{0} & \alert{0} & \alert{1} & \alert{|} & \raa{\alert{1}} & \alert{0} & \alert{0} & \alert{\ldots}
\end{array}$$

}

\frame{
\frametitle{$W_{T_2}$}

$$\begin{array}{ccccccccc}
\ldots & 0 & 0 & 0 & | & \raa{0} & 0 & 0 & \ldots \\
\ldots & 0 & 0 & \laa{0} & | & 1 & 0 & 0 & \ldots \\
\ldots & 0 & 0 & 1 & | & \raa{1} & 0 & 0 & \ldots \\
\ldots & 0 & 0 & 1 & | & 2 & \raa{0} & 0 & \ldots \\
\ldots & 0 & 0 & 1 & | & \laa{2} & 1 & 0 & \ldots \\
\ldots & 0 & 0 & \laa{1} & | & 0 & 1 & 0 & \ldots \\
\ldots & 0 & \laa{0} & 2 & | & 0 & 1 & 0 & \ldots \\
\ldots & 0 & 1 & \raa{2} & | & 0 & 1 & 0 & \ldots \\
\ldots & 0 & 1 & 0 & | & \raa{0} & 1 & 0 & \ldots \\
\ldots & 0 & 1 & \laa{0} & | & 1 & 1 & 0 & \ldots \\
\alert{\ldots} & \alert{0} & \alert{1} & \alert{1} & | & \raa{\alert{1}} & \alert{1} & \alert{0} & \alert{\ldots}
\end{array}$$

}

\frame{
\frametitle{Proof of Lemma: Base Case}

\begin{proof}
By induction on $n$.

\pause

Base case: we have already seen $W_{T_1}$ (and $W_{T_2}$).
\end{proof}
}

\frame{
\frametitle{Proof of Lemma: Inductive Step}

Assume:
\begin{enumerate}
\item The world $W_{T_{n-1}}$ has the form: $\ldots 0 1^{n-1} | \raa{1} 1^{n-2} 0 \ldots$
\item The progression from $W_0$ to $W_{T_{n-1}}$ does not step outside of the locations between $-(n-1)$ and $n-2$.
\end{enumerate}

}

\frame{
\frametitle{Proof of Lemma: Inductive Step (cont.)}

\begin{proof}

$$\begin{array}{ccccc|ccccc}
W_0: & \ldots & 0 & 0 & 0^{n-1} & \raa{0} & 0^{n-2} & 0 & 0 & \ldots \\ \pause
\downarrow \\
W_{T_{n-1}}: & \ldots & 0 & 0 & 1^{n-1} & \raa{1} & 1^{n-2} & 0 & 0 & \ldots \\ \pause
\downarrow \\
W_{T_{n-1} + n - 1}: & \ldots & 0 & 0 & 1^{n-1} & 2 & 2^{n-2} & \raa{0} & 0 & \ldots \\ \pause
\downarrow \\
W_{T_{n-1} + 3n - 2}: & \ldots & 0 & \laa{0} & 2^{n-1} & 0 & 0^{n-2} & 1 & 0 & \ldots \\ \pause
\downarrow \\
W_{T_{n-1} + 4n - 2}: & \ldots & 0 & 1 & 0^{n-1} & \raa{0} & 0^{n-2} & 1 & 0 & \ldots \\ \pause
\downarrow \\
W_{2T_{n-1} + 4n - 2}: & \ldots & 0 & 1 & 1^{n-1} & \raa{1} & 1^{n-2} & 1 & 0 & \ldots \\ \pause
\alert{=W_{T_n}}
\end{array}$$

\end{proof}
}

\frame{
\frametitle{Recurrence for $T_n$}

\begin{corollary}
$T_n = 2T_{n-1}+4n-2$.
\end{corollary}

}

\frame{
\frametitle{Solving the Recurrence}

$T_n = 2T_{n-1}+4n-2$

\pause
\vspace{.25in}

Assume $T_n = x \cdot 2^n+yn+z$ for some $x,y,z$ and solve.

\pause

\alert{
\begin{align*}
T_n &= 6 \cdot 2^n - 4n - 6 \\
T_n &= \Theta(2^n).
\end{align*}
}

}

\frame {
\frametitle{Further work}

\begin{columns}
\column{0.65\textwidth}
\begin{itemize}
\item Other complete rules
\pause
\begin{itemize}
\item $k$-complete $k$-rule?
\end{itemize}
\pause
\item Boundedness
\pause
\end{itemize}
\column{0.35\textwidth}
\includegraphics{flea.jpg}
\end{columns}

}

\end{document} 